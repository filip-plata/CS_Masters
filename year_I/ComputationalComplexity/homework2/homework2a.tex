\documentclass{article}

\usepackage{fancyhdr}
\usepackage{extramarks}
\usepackage{amsmath}
\usepackage{amsthm}
\usepackage{amsfonts}
\usepackage{tikz}
\usepackage[plain]{algorithm}
\usepackage{algpseudocode}

\usetikzlibrary{automata,positioning}

%
% Basic Document Settings
%

\topmargin=-0.45in
\evensidemargin=0in
\oddsidemargin=0in
\textwidth=6.5in
\textheight=9.0in
\headsep=0.25in

\linespread{1.1}

\pagestyle{fancy}
\lhead{\hmwkAuthorName}
\chead{\hmwkClass\ : \hmwkTitle}
\rhead{\firstxmark}
\lfoot{\lastxmark}
\cfoot{\thepage}

\renewcommand\headrulewidth{0.4pt}
\renewcommand\footrulewidth{0.4pt}

\setlength\parindent{0pt}

%
% Create Problem Sections
%

\newcommand{\enterProblemHeader}[1]{
    \nobreak\extramarks{}{Problem \arabic{#1} continued on next page\ldots}\nobreak{}
    \nobreak\extramarks{Problem \arabic{#1} (continued)}{Problem \arabic{#1} continued on next page\ldots}\nobreak{}
}

\newcommand{\exitProblemHeader}[1]{
    \nobreak\extramarks{Problem \arabic{#1} (continued)}{Problem \arabic{#1} continued on next page\ldots}\nobreak{}
    \stepcounter{#1}
    \nobreak\extramarks{Problem \arabic{#1}}{}\nobreak{}
}

\setcounter{secnumdepth}{0}
\newcounter{partCounter}
\newcounter{homeworkProblemCounter}
\setcounter{homeworkProblemCounter}{1}
\nobreak\extramarks{Problem \arabic{homeworkProblemCounter}}{}\nobreak{}

%
% Homework Problem Environment
%
% This environment takes an optional argument. When given, it will adjust the
% problem counter. This is useful for when the problems given for your
% assignment aren't sequential. See the last 3 problems of this template for an
% example.
%
\newenvironment{homeworkProblem}[1][-1]{
    \ifnum#1>0
        \setcounter{homeworkProblemCounter}{#1}
    \fi
    \section{Problem \arabic{homeworkProblemCounter}}
    \setcounter{partCounter}{1}
    \enterProblemHeader{homeworkProblemCounter}
}{
    \exitProblemHeader{homeworkProblemCounter}
}

%
% Homework Details
%   - Title
%   - Due date
%   - Class
%   - Section/Time
%   - Instructor
%   - Author
%

\newcommand{\hmwkTitle}{Homework\ \#2}
\newcommand{\hmwkDueDate}{December 19, 2019}
\newcommand{\hmwkClass}{Computational Complexity}
\newcommand{\hmwkAuthorName}{\textbf{Filip Plata}}

%
% Title Page
%

\title{
    \vspace{2in}
    \textmd{\textbf{\hmwkClass:\ \hmwkTitle}}\\
    \normalsize\vspace{0.1in}\small{Due\ on\ \hmwkDueDate\ at 23:59}\\
    \vspace{3in}
}

\author{\hmwkAuthorName}
\date{\today}

\renewcommand{\part}[1]{\textbf{\large Part \Alph{partCounter}}\stepcounter{partCounter}\\}

%
% Various Helper Commands
%

% Useful for algorithms
\newcommand{\alg}[1]{\textsc{\bfseries \footnotesize #1}}

% For derivatives
\newcommand{\deriv}[1]{\frac{\mathrm{d}}{\mathrm{d}x} (#1)}

% For partial derivatives
\newcommand{\pderiv}[2]{\frac{\partial}{\partial #1} (#2)}

% Integral dx
\newcommand{\dx}{\mathrm{d}x}

% Alias for the Solution section header
\newcommand{\solution}{\textbf{\large Solution}}

% Probability commands: Expectation, Variance, Covariance, Bias
\newcommand{\E}{\mathrm{E}}
\newcommand{\Var}{\mathrm{Var}}
\newcommand{\Cov}{\mathrm{Cov}}
\newcommand{\Bias}{\mathrm{Bias}}

\begin{document}

\maketitle

\pagebreak

\begin{homeworkProblem}[1]

	\textbf{Initial remarks}
	
	We will refer to the problem from task as language L in this solution. We will also
	say that sequence of letters (possibly finite) is \textit{correct}, if it satisfies
	all constraints from tuples. This assumes we will always have some implicit alphabet
	and tuple set throught the solution and we will omit referring to it.
	\\

    \textbf{Main Solution - $L \in \textsc{PSPACE}$}

	We will start by proving that $L \in \textsc{PSPACE}$. We will use the fact that
	$\textsc{PSPACE} = \textsc{NPSPACE}$, so it is enough to prove that 
	$L \in \textsc{NPSPACE}$. Let us by C denote maximum number of positions to the right
	that a rule can influence, e.g. the maximum of $j + k + 1$ over all tuples. Our
	algorithm will guess letters one by one, always keeping in memory previous C letters
	(at the beginning, it starts with $C-1$ blanks which are not in alphabet and $x_0$).
	After every guess it checks if rules would hold after appending new letter, and then
	shifts letters by one and inserts guessed letter. If no letter can be appended, 
	the algorithm rejects. 
	We also keep a counter, increasing it by one with every new letter. When
	counter will reach a value $> 3 * (|A| + 1)^{C}$, where $|A|$ is alphabet size, 
	we accept
	- we know we have a cycle in memorized letters, and machine could just repeat it
	\textit{ad infinitum}. For counter we need linear memory with respect to C, which
	is linear (so also polynomial) in terms of tuple sizes. If algorithm cannot reach
	our counter threshold, then there is no correct sequence of such length, which
	means also there is no infinite sequence.
	\\
	
	\textbf{Main Solution - L in \textsc{PSPACE-complete}}
	
	We will perform reduction from TILING problem. This means we have a set of tiles
	$T$, tiles $t_1, t_2 \in T$, a number $n_x$ written in unary and a set of rules,
	which
	decide whether two tiles can be neighbours. We ask whether there exists $n_y$,
	such that a grid $n_x \times n_y$ can be covered by tiles, such that $t_1$ is in 
	lower left corner, and $t_2$ is in upper right corner and all neighbours fit to each
	other. This problem (from tutorials) was proven to be \textsc{PSPACE-complete}.
	\\
	
	We will begin by high level overview of the reduction. We linearize grid, by
	putting consecutive rows next to each other. Next, we will force every correct
	sequence to take form of batches of letters of fixed size next to each other. In
	those batches, we will keep administrative information - index of the batch 
	(in binary), one row of tiles and information whether index has overflown -
	this will inform us to disallow such sequences. Using tuples
	we will enforce neighbouring rules on tiles. We will also have in batch a letter
	at its very beginning, to mark that it is a batch. If the tile in last row is
	the one we want, we will allow the sequence to transform into infinite stream of
	one special letter marking success.
	\\
	
	The idea comes from the fact, that if there is a correct tiling, then because total
	number of distinct rows is not greater than $|T|^{n_x}$, a solution can be made
	shorter than $2 \cdot |T|^{n_x} + 1$. This means that if our sequence gets too long,
	we can try to introduce a contradicting rules. So if there is a tiling, our letter
	sequence will have to escape into special success symbol - which will be possible
	only after finding correct upper right tile. Conversly, if there is a infinite
	sequence, this means at some point it contains success symbols only, and we can
	extract tiling from consecutive batches. The reader can note that with this approach
	there are two orthogonal problems - accepting only valid tilings, and then forcing
	a sequence to be possibly infinite only when correct tiling is found. Our 
	administrative information - counter and overflow marker - serves the purpose to
	forcibly end the 'computation' of sequence.
	\\
	
	Now a formal description of reduction. Firstly, as alphabet
	we will use 
	
	\[ A = T \cup \{ OK, NO, NO', ({0, 1}, \{ OVERFLOW, NO\_OVERFLOW \}), BATCH, BATCH\_INI \} \] 
	notation is not formal, we expand every possibility for bits
	and overflow marker into four letters.
	This means additionaly to tiles, we will have 'OK' letter - success symbol - NO and
	NO' will serve to introduce contradicting rules - using contradicting rules, we can
	force correct sequences to have some properties. $BATCH\_INI$ is a letter $x_0$, we
	will use it to initialze administrative information and lower left tile. BATCH
	represents any consecutive batch before success has been reached. The idea is to
	force BATCH letters at the beggining of batch to have relative marker for rules.
	Now we will proceed to description of rules. This will be done in stages, for each
	part of batch seperately. Now we will define some symbols for lengths and offsets
	inside a batch.
	Let the length of counter be $CTR\_LEN$ - it has to hold values on the order of
	$|T|^{n_x}$, so it requires linear space with respect to $n_x$ - since $n_x$ is
	written in unary, and we write counter in binary. Let offset of tiles be $TIL\_OFF$.
	And now we will define some macro rules. This will be gadgets we will use to shorten
	the formal description:
	
	\begin{itemize}

		\item DISALLOW(x, k, y) = (x, k, y, 1, NO), (x, k, y, 1, NO') - disallow x followed by y k letters latter - no correct infinite sequence can contain such two letters wiht those two rules
		\item ENFORCE(x, k, S) = DISALLOW(x, k, $s'$) for $s' \in A \setminus S $, where S is some set of letters
	
	\end{itemize}
	
	We will also use sets of letters in macro rules. This we want to expand to set of rules, with
	every possible combination of letters from sets, e.g: 
	
	DISALLOW($\{ a, b \}$, k, c) = DISALLOW(a, k, c) and DISALLOW(b, k, c)
	
	And now on to stages. The layout of a batch will be:
	
	\begin{itemize}
	
		\item BATCH marker
		\item Counter - a fixed length number of bits
		\item tiles
		
	\end{itemize}		
	
	\textbf{Initial batch}
	We introduce following rules:
	
	\begin{itemize}
		\item (BATCH\_INI, 0, BATCH\_INI, k, (0, NO\_OVERFLOW) ) for $1 \leq k \leq CTR\_LEN$ - setting counter to zero
		\item ENFORCE(BATCH\_INI, BATCH\_LEN, $\{ BATCH, OK \}$) - we could be lucky and finish with first row only
		\item ENFORCE(BATCH\_INI, TIL\_OFF, $\{ t_1 \}$) - initial tiling letter
	\end{itemize}
	
	\textbf{Incrementing counter by one}
	
	We force initialization of counter only if there is a next BATCH. Then four addition
	rules, that force counter incremented by one in the next batch - if a counter was
	forced. Then there is a rules to copy counter bits after incrementing stops at some
	position.
	
	\begin{itemize}
	
		\item ( (1, \_), $BATCH\_LEN - 1$, BATCH, 1, (0, OVERFLOW))
		\item ( (0, \_), $BATCH\_LEN - 1$, BATCH, 1, (1, NO\_OVERFLOW))
		
		\item ( (1, \_), $BATCH\_LEN - 1$, (0, OVERFLOW), 1,  (0, OVERFLOW))
		\item ( (0, \_), $BATCH\_LEN - 1$, (0, OVERFLOW), 1,  (1, NO\_OVERFLOW))
		\item ( (0, \_), $BATCH\_LEN - 1$, (1, OVERFLOW), 1,  (0, OVERFLOW))
		\item ( (1, \_), $BATCH\_LEN - 1$, (1, OVERFLOW), 1,  (1, OVERFLOW))
		\item ( (bit, \_), $BATCH\_LEN - 1$, (\_, NO\_OVERFLOW), 1,  (bit, NO\_OVERFLOW))
	\end{itemize}
	
	Effectively, if there is a next batch marker, those rules will increment it by one
	in the next batch - meaning every correct sequence must have such behaviour.
	\\
	
	\textbf{Disallowing long inproductive sequences}
	
	DISALLOW(BATCH, CTR\_LEN, (1, \_)) - if counter highest bit ever reaches 1, such
	sequence cannot continue. This gives us that if there exists an infinite sequence,
	it must at some point escape calculating consecutive batches into OK stream.
	\\
	
	\textbf{Enforcing tiling constraints}
	
	\begin{itemize}
		\item ENFORCE(t, 1, GOOD\_RIGHT(t)) for $t \in T$ - only allow correct right neighbours from tiling to appear as next letter to a tile. BATCH and OK are also fine
		\item ENFORCE(t, $BATCH\_LEN$, GOOD\_UP(t)) - we also allow BATCH and OK
		\item ENFORCE($\{ BATCH, BATCH\_INI \}$, l, T), for $TIL\_OFF \leq l \le BATCH\_LEN - 1$ - we need aditional constraint to allow only tiling letters on $n_x$ consecutive positions. It looks contrived, but we do not want to disallow for the last tile in the row to be next to BATCH or OK
	\end{itemize}
	
	\textbf{Computation flow - do we end or not}
	
	\begin{itemize}
		\item ($\{BATCH, BATCH\_INI\}$, $TIL\_OFF + n_x - 1$, $t_2$, 1, OK) - we end computation when we see correct last tile
		\item ($\{BATCH, BATCH\_INI\}$, $TIL\_OFF + n_x - 1$, $T \setminus \{ t_2 \}$, 1, BATCH) - so we can place OK only if last tile is $t_2$. This is $2 * (|T| - 1)$ rules
		\item (OK, 0, OK, 1, OK) - we truly end computation, only OK allowed after first OK, thus we easily have infinite sequence
	\end{itemize}
	
	\textbf{Proof of equivalence}
	
	If there exists a correct tiling, we can construct a sequence adhering to such rules:
	counter for consequtive batches is incresed by one, and for tiles we use tiles from
	tiling, row by row. After last row we have infinite sequence of OK letters. Such
	sequence satisfies all constraints. Conversly, if we have an infinite sequence, this
	means we never encounter 1 on highest bit of counter. Assuming our sequence does not
	contain a $t_2$ tile at the end of a batch, we would never see any OK letter. This
	means that we have BATCH symbols, which enforce incrementing counter in every batch,
	and at some point a contradictory rules for highest bit of counter set to one
	would have to be enforced, which is a contradiction (meaning - our sequence cannot
	be infinite, because it cannot continue). This means we do see an OK letter, which
	means we do have a matching $t_2$ tile, and we could extract tiles from batches
	to form a tiling.
	\\
	Thus, the languages are equivalent. One can easily see, that to perform reduction
	it suffices to use $O(log (n_x))$ memory - every macro can be expanded separately,
	and additional logarihmic memory is enough to expand each of them. This completes
	the proof. The number of macros is $O(n_x)$, each has size also $O(n_x)$, thus the
	memory used is polynomial ($O(n_x ^ 2)$) in terms of $n_x$.
	

\end{homeworkProblem}

\pagebreak



\pagebreak

\end{document}
