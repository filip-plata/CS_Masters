\documentclass{article}

\usepackage{fancyhdr}
\usepackage{extramarks}
\usepackage{amsmath}
\usepackage{amsthm}
\usepackage{amsfonts}
\usepackage{tikz}
\usepackage[plain]{algorithm}
\usepackage{algpseudocode}

\usetikzlibrary{automata,positioning}

%
% Basic Document Settings
%

\topmargin=-0.45in
\evensidemargin=0in
\oddsidemargin=0in
\textwidth=6.5in
\textheight=9.0in
\headsep=0.25in

\linespread{1.1}

\pagestyle{fancy}
\lhead{\hmwkAuthorName}
\chead{\hmwkClass\ : \hmwkTitle}
\rhead{\firstxmark}
\lfoot{\lastxmark}
\cfoot{\thepage}

\renewcommand\headrulewidth{0.4pt}
\renewcommand\footrulewidth{0.4pt}

\setlength\parindent{0pt}

%
% Create Problem Sections
%

\newcommand{\enterProblemHeader}[1]{
    \nobreak\extramarks{}{Problem \arabic{#1} continued on next page\ldots}\nobreak{}
    \nobreak\extramarks{Problem \arabic{#1} (continued)}{Problem \arabic{#1} continued on next page\ldots}\nobreak{}
}

\newcommand{\exitProblemHeader}[1]{
    \nobreak\extramarks{Problem \arabic{#1} (continued)}{Problem \arabic{#1} continued on next page\ldots}\nobreak{}
    \stepcounter{#1}
    \nobreak\extramarks{Problem \arabic{#1}}{}\nobreak{}
}

\setcounter{secnumdepth}{0}
\newcounter{partCounter}
\newcounter{homeworkProblemCounter}
\setcounter{homeworkProblemCounter}{1}
\nobreak\extramarks{Problem \arabic{homeworkProblemCounter}}{}\nobreak{}

%
% Homework Problem Environment
%
% This environment takes an optional argument. When given, it will adjust the
% problem counter. This is useful for when the problems given for your
% assignment aren't sequential. See the last 3 problems of this template for an
% example.
%
\newenvironment{homeworkProblem}[1][-1]{
    \ifnum#1>0
        \setcounter{homeworkProblemCounter}{#1}
    \fi
    \section{Problem \arabic{homeworkProblemCounter}}
    \setcounter{partCounter}{1}
    \enterProblemHeader{homeworkProblemCounter}
}{
    \exitProblemHeader{homeworkProblemCounter}
}

%
% Homework Details
%   - Title
%   - Due date
%   - Class
%   - Section/Time
%   - Instructor
%   - Author
%

\newcommand{\hmwkTitle}{Homework\ \#2b}
\newcommand{\hmwkDueDate}{January 13, 2019}
\newcommand{\hmwkClass}{Computational Complexity}
\newcommand{\hmwkAuthorName}{\textbf{Filip Plata}}

%
% Title Page
%

\title{
    \vspace{2in}
    \textmd{\textbf{\hmwkClass:\ \hmwkTitle}}\\
    \normalsize\vspace{0.1in}\small{Due\ on\ \hmwkDueDate\ at 23:59}\\
    \vspace{3in}
}

\author{\hmwkAuthorName}
\date{\today}

\renewcommand{\part}[1]{\textbf{\large Part \Alph{partCounter}}\stepcounter{partCounter}\\}

%
% Various Helper Commands
%

% Useful for algorithms
\newcommand{\alg}[1]{\textsc{\bfseries \footnotesize #1}}

% For derivatives
\newcommand{\deriv}[1]{\frac{\mathrm{d}}{\mathrm{d}x} (#1)}

% For partial derivatives
\newcommand{\pderiv}[2]{\frac{\partial}{\partial #1} (#2)}

% Integral dx
\newcommand{\dx}{\mathrm{d}x}

% Alias for the Solution section header
\newcommand{\solution}{\textbf{\large Solution}}

% Probability commands: Expectation, Variance, Covariance, Bias
\newcommand{\E}{\mathrm{E}}
\newcommand{\Var}{\mathrm{Var}}
\newcommand{\Cov}{\mathrm{Cov}}
\newcommand{\Bias}{\mathrm{Bias}}

\begin{document}

\maketitle

\pagebreak

\begin{homeworkProblem}[4]

    \textbf{Solution}
    
    We will describe machine using logarithmic memory and the circuit it will produce.
    Given input word, machine will start with calculating n in binary (by adding one to binary counter every time it sees a symbol one, while reading input symbol by symbol) and then with calculating m, the length of subwords. It
    is defined by equality $n = 2^m \cdot m$, where n is input word length. This can
    be done in logarithmic space wrt. $n$, by dividing input $n$ in a loop by two (simple shift, one operation),
    and counting the number of divisions. At each step, the algorithm checks if reminder
    is equal to the count of divisions, and if so, $m$ is equal to the counter at this
    point. If we cannot divide number by two or equality of counter and reminder is never
    reached, the input word length is incorrect and we can print a circuit returning 
    "NO", for every word (for instance, AND of all bits including negated ones).
    \\
    
    After obtaining $m$, we will have another machine working in logarithmic space. It
    was proven, that composition of two logarithmic-space machine can be done in
    logarithmic space (also, m is $O(log(n))$, so there is enough space to remember everything). For every distinct pair of subwords,
    we will create a circuit 
    which will check for their inequality, and then a circuit which will AND all
    previous outputs. This clearly will solve \textit{Perm} language.
    \\
    
    To do so, machine will require two counters i, j - each of them using logarithmic 
    space,
    which will mark which subword we are on. If they are equal, we skip. If they are
    different, we generate a circuit named $\#(i, j)$, which returns whether words
    $w_i, w_j$ are different. Then, we generate an AND gate, taking every $\#(i, j)$ as
    input. Circuit for word difference can be done as bitwise XOR, and then an OR gate
    taking $m$ inputs. Printing it might require a third counter, counting up to m.
    When all bits are equal, this will return FALSE, and if any bit pair is distinct, it
    will return TRUE, which is what we want. XOR gate was done during classes from basic
    gates. Generated circut has a depth of $1 + 1 + depth(XOR)$, which is constant.
    \\
    
    Thus, \textit{Perm} is log-space uniform $AC^0$.

	

\end{homeworkProblem}

\pagebreak



\pagebreak

\end{document}
