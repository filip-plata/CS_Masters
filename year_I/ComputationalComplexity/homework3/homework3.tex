\documentclass{article}

\usepackage{fancyhdr}
\usepackage{extramarks}
\usepackage{amsmath}
\usepackage{amsthm}
\usepackage{amsfonts}
\usepackage{tikz}
\usepackage[plain]{algorithm}
\usepackage{algpseudocode}

\usetikzlibrary{automata,positioning}

%
% Basic Document Settings
%

\topmargin=-0.45in
\evensidemargin=0in
\oddsidemargin=0in
\textwidth=6.5in
\textheight=9.0in
\headsep=0.25in

\linespread{1.1}

\pagestyle{fancy}
\lhead{\hmwkAuthorName}
\chead{\hmwkClass\ : \hmwkTitle}
\rhead{\firstxmark}
\lfoot{\lastxmark}
\cfoot{\thepage}

\renewcommand\headrulewidth{0.4pt}
\renewcommand\footrulewidth{0.4pt}

\setlength\parindent{0pt}

%
% Create Problem Sections
%

\newcommand{\enterProblemHeader}[1]{
    \nobreak\extramarks{}{Problem \arabic{#1} continued on next page\ldots}\nobreak{}
    \nobreak\extramarks{Problem \arabic{#1} (continued)}{Problem \arabic{#1} continued on next page\ldots}\nobreak{}
}

\newcommand{\exitProblemHeader}[1]{
    \nobreak\extramarks{Problem \arabic{#1} (continued)}{Problem \arabic{#1} continued on next page\ldots}\nobreak{}
    \stepcounter{#1}
    \nobreak\extramarks{Problem \arabic{#1}}{}\nobreak{}
}

\setcounter{secnumdepth}{0}
\newcounter{partCounter}
\newcounter{homeworkProblemCounter}
\setcounter{homeworkProblemCounter}{1}
\nobreak\extramarks{Problem \arabic{homeworkProblemCounter}}{}\nobreak{}

%
% Homework Problem Environment
%
% This environment takes an optional argument. When given, it will adjust the
% problem counter. This is useful for when the problems given for your
% assignment aren't sequential. See the last 3 problems of this template for an
% example.
%
\newenvironment{homeworkProblem}[1][-1]{
    \ifnum#1>0
        \setcounter{homeworkProblemCounter}{#1}
    \fi
    \section{Problem \arabic{homeworkProblemCounter}}
    \setcounter{partCounter}{1}
    \enterProblemHeader{homeworkProblemCounter}
}{
    \exitProblemHeader{homeworkProblemCounter}
}

%
% Homework Details
%   - Title
%   - Due date
%   - Class
%   - Section/Time
%   - Instructor
%   - Author
%

\newcommand{\hmwkTitle}{Homework\ \#3}
\newcommand{\hmwkDueDate}{January 27, 2020}
\newcommand{\hmwkClass}{Computational Complexity}
\newcommand{\hmwkAuthorName}{\textbf{Filip Plata}}

%
% Title Page
%

\title{
    \vspace{2in}
    \textmd{\textbf{\hmwkClass:\ \hmwkTitle}}\\
    \normalsize\vspace{0.1in}\small{Due\ on\ \hmwkDueDate\ at 23:59}\\
    \vspace{3in}
}

\author{\hmwkAuthorName}
\date{\today}

\renewcommand{\part}[1]{\textbf{\large Part \Alph{partCounter}}\stepcounter{partCounter}\\}

%
% Various Helper Commands
%

% Useful for algorithms
\newcommand{\alg}[1]{\textsc{\bfseries \footnotesize #1}}

% For derivatives
\newcommand{\deriv}[1]{\frac{\mathrm{d}}{\mathrm{d}x} (#1)}

% For partial derivatives
\newcommand{\pderiv}[2]{\frac{\partial}{\partial #1} (#2)}

% Integral dx
\newcommand{\dx}{\mathrm{d}x}

% Alias for the Solution section header
\newcommand{\solution}{\textbf{\large Solution}}

% Probability commands: Expectation, Variance, Covariance, Bias
\newcommand{\E}{\mathrm{E}}
\newcommand{\Var}{\mathrm{Var}}
\newcommand{\Cov}{\mathrm{Cov}}
\newcommand{\Bias}{\mathrm{Bias}}

\begin{document}

\maketitle

\pagebreak

\begin{homeworkProblem}[5]

    \textbf{Main Solution}
    \\
    
    We will describe a recursive algorithm, that will solve the
    problem in polynomial time. We will uses polynomial primality test as
    a black box, and we will use SAT solver to extract some nontrivial factorization
    from composite numbers.
    \\
    
    We are given a number $n$ with $k$ bits and $m$ primes factors in total (summing all 
    the exponents). 
    Let us denote by $T(n)$ the highest
    possible number of factorization subroutine calls, where at each level we 
    will obtain some
    non-trivial (so no one times whole number) factorization. We will later prove that 
    $T(n) \leq m - 1$, thus if obtaining any non-trivial factorization could be done
    in polynomial time, the whole factorization can be done in such time.
    \\
    
    We will begin with more formal description of the algorithm. We shall start with
    a polynomial algorithm for obtaining non-trivial factorization. We assume our
    input is a composite number $n$, and we want to find $a \cdot b = n$, such that
  	$a \neq 1, b \neq 1$. Let us again by $k$ denote number of bits in $n$. Let us
  	imagine multiplication of two $k-bit$ numbers - $x_1, ..., x_k$ and $y_1, ..., y_k$
  	- and it at most $2k-bit$ result $z_1, ..., z_{2k}$. We will construct a formula $\Phi$,
  	that will be satisfiable only when $x \cdot y = n, x \neq 1, y \neq 1$. 
  	We only need $\Phi$ to be polynomial in size wrt. $k$ - then using oracle, a possible
  	factorization can be guessed one bit at a time, resulting in polynomial algorithm
  	which outputs some factorization into two factors.
  	On the other hand, because $n$ is composite, we know $\Phi$ can be satisfied. We shall
  	change perspective a bit. We know from classes, that there is a polynomial size
  	circuit, using gates with binary input, that performs multiplication. Thus, for every
  	bit of output, we can extract a boolean formula describing it, which size is less
  	than total number of gates - which means also polynomial. So we can construct such
  	formula $\Phi$, by having a formula for every bit of $z$, with additional few operations
  	to ensure $z_i = n_i$ ($i-th$ bit of $n$, this could be done as xor). Then, we need
  	to perform an `and` with formulas which exclude ones, to obtain $\Phi$.
  	\\
  	
  	We have a polynomial algorithm for finding a non-trivial factorization for composite
  	numbers. Our main algorithm will be recursive, outputting a list of factors and powers
  	(output will be polynomial wrt. input). It will first check if input is a prime. If so,
  	we have a factorization. If not, it will call factorization primitve described above.
  	Then, it will call recursively for each factor, and merge respective output lists.
  	Merging of the lists can be done in quadratic time (on Turing machine), because they
  	are sorted. So the main concern for complexity are recursive calls and their number.
  	We will prove inequality from the beginning, that $T(n) \leq m - 1$, where m is total
  	count of prime factors for number $n$. This will be by induction, assuming it is true
  	for every smaller number then some threshold $n$. For prime numbers, it is obvious, we do
  	not call factorization. For a compound number, after calling factorization we get two numbers
  	$x, y$, smaller then $n$. Thus, by induction we know that $T(x) \leq m_x - 1$ and
  	$T(y) \leq m_y - 1$, where $m_x, m_y$ are respective totals of prime factors for $x, y$.
  	Because $m_x + m_y = m$, we have:

	\[ T(n) \leq 1 + T(x) + T(y) \leq 1 + (m_x - 1) + (m_y - 1) = m - 1 \]
	
	which concludes the prove of inequality.
	
	It means our algorithm calls factorization subroutine at most this many times, as it has
	prime factors. Number of prime factors is bounded by $\log_2 n \leq k + 1$.
	Then, it calls prime checking
	subroutine also at most this many times - because each time this is done we get new 
	prime factor. 
	Also merging is done once for every factorization subroutine call.
	Because every of those subroutines can be done in polynomial time wrt. $k$, the whole
	algorithm is k times slower than sum of those three complexity bounds, which is still
	a polynomial.
	\\
	
	This concludes the proof.


\end{homeworkProblem}

\pagebreak

\begin{homeworkProblem}[6]

	\textbf{Common remarks}
	
	Let us by $B(w, r)$ denote a ball in metric
	space of words with Hamming distance. We will provide a bound for the number of 
	words in such ball for a word $w$ and radius $r$. We 
	begin by noting we can choose at most $|w|\choose r$ positions, and then any subset
	of which we have $2^r$, to arrive at any possible word in $B(w, r)$. We may count
	each word many times, but we only need an upper bound. After multiplication
	we arrive at the result that size of $B(w, r)$ is polynomial wrt. $|w|$,
	because $r$ is a constant.
	\\
	
	We will assume language $L$ and number $r$ ar given from now on.
	\\
	
	\textbf{Main Solution for a)}
	\\
	
	Now we will show an algorithm which will answer queries for language $B(L, r)$. Given
	a word $w$, for every word in $B(w, r)$ we run algorithm for language L, obtaining
	some answers. If we ever see a YES answer, we know some word in a ball is in L,
	thus via symmetry of distance our word w is in $B(L, r)$. If we only see NO, we
	return NO. Thus, if $w \notin B(L, r)$, we can only see NO answers on elements of
	$B(w, r)$ and our algorithm will always return NO, as we require of algorithms from
	$\mathcal{RP}$. On the other hand if $w \in B(L, r)$, we know there is at least one
	word in the ball, for which we had the probability of $\frac{1}{2}$ to return YES.
	So we can bound the probablity for YES as at least $\frac{1}{2 \cdot | B(w, r) |}$.
	Our algorithm runs a polynomial algorithm polynomially many times, which means it
	is still polynomial. To increase the probability, we run the combined algorithm 
	$\Theta(|B(w, r)|)$ times. If we run algorithm $u$ times, we achive probability of
	$1 - (1 - \frac{1}{2 \cdot | B(w, r) |}) ^ {u}$. From Bernoulli inequality we obtain:

	\[ (1 - \frac{1}{2 \cdot | B(w, r) |}) ^ {u} \leq \frac{1}{1 + \frac{u}{2 \cdot | B(w, r) |}} \]
	\[ 1 - (1 - \frac{1}{2 \cdot | B(w, r) |}) ^ {u} \geq 1  - \frac{1}{1 + \frac{u}{2 \cdot | B(w, r) |}} \]
	
	Which means it is sufficient to take $u = 2 \cdot |B(w, r)|$ to surpass the
	threshold of $\frac{1}{2}$. Our initial remark was that $|B(w, r)|$ is a polynomial
	wrt. $|w|$, which means we only need again to run combined algorithm polynomial
	number of times to reach required accuracy. This gives us a polynomial algorithm
	for $B(L, r)$, which has large enough probability threshold, which means that
	$B(L, r) \in \mathcal{RP}$.
	\\
	
	\textbf{Main Solution for b)}
	\\
	
	We will use similar idea as in \textbf{a)}. We will describe an algorithm for $B(L, r)$, proving it is in $co\mathcal{RP}$. For every word in ball $B(w, r)$, we run
	algorithm for language $L$ for $k$ times. We look at results for each word separately.
	If for some word $w'$ we see at least one NO, we know $w' \notin L$. If every result is YES for $w'$,
	we assume $w' \in L$, and thus $w \in B(L, r)$. If for every $w' \in B(w, r)$ we obtain
	have obtained at least one NO, we know for sure that $w \notin B(L, r)$. Under this scheme, if $w \in B(L, r)$,
	there is a $w' \in L$ in the ball, which will return always YES, and thus we will always
	assign YES as output of the main algorithm. If $w \notin B(L, r)$, assuming independence of algorithm runs,
	we get a probability of NO at least:
	
	\[ (1 - (\frac{1}{2}) ^ k) ^ {|B(w, r)|} \geq 1 - \frac{|B(w, r)|}{2^k} \]
	
	where we have used the Bernoulli inequality. This means $k$ of the order of $|B(w, r)|$
	is more then enough to achieve probability of at least $\frac{1}{2}$. Because we require
	$k \cdot |B(w, r)|$ runs of a polynomial algorithm, the combined algorithm is still polynomial
	in time. Thus, $B(L, r) \in co\mathcal{RP}$, as required.

\end{homeworkProblem}


\pagebreak

\end{document}
